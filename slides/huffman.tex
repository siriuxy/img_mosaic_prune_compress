\documentclass[table]{beamer}

\usepackage{pgfpages}
\pgfpagesuselayout{resize to}[a4paper,landscape,border shrink=5mm]

\usetheme{Luebeck}
\usepackage[utf8]{inputenc}

\title[CS225 Online Video Series: Huffman]{CS225 Online Video Series:
    Huffman Encoding}
\author{Chase Geigle}
\date{\today}
\institute{University of Illinois at Urbana-Champaign}

\begin{document}
    \frame{\titlepage}

    \section{Overview}
    \frame{\tableofcontents}

    \section{Motivation}
    \begin{frame}
        \frametitle{Motivation}
        \begin{itemize}
            \item<2-> Compression is important:
            \begin{itemize}
                \item<3-> Improve ability to transfer over a network
                (bandwidth may be expensive, connections may be slow, etc)
                \item<4-> Reduce storage usage
            \end{itemize}
            \item<5-> Types of compression:
            \begin{itemize}
                \item<6-> \emph{Lossy} encoding
                \begin{itemize}
                    \item<7-> JPEG is an example
                    \item<8-> Not guaranteed to get the same output
                    bit-for-bit when you decompress!
                \end{itemize}
                \item<9-> \emph{Lossless} encoding
                \begin{itemize}
                    \item<10-> ZIP and TGZ are examples
                    \item<11-> PNG too!
                    \item<12-> \textbf{Guaranteed} to get the \emph{same
                    output} bit-for-bit when you decompress!
                    \item<13-> Huffman coding is a form of lossless
                    compression.
                \end{itemize}
            \end{itemize}
        \end{itemize}
    \end{frame}

    \section{Implementation}
    \begin{frame}
        \frametitle{Intuition}
        \begin{itemize}
            \item<2-> Say we have some text file.
            \item<3-> There is some kind of redundant (or repeated)
            information in this file.
            \item<4-> Moreover, the repetition is uneven (non-uniform) in
            distribution.
            \item<5-> \textbf{Key Idea:} If we allocate a \textbf{small
            number} of bits to frequent things, and a larger number of bits
            to infrequent things, we can compress the file.
        \end{itemize}
    \end{frame}

    \begin{frame}
        \frametitle{Algorithm}
        \begin{itemize}
            \item<2-> Determine how frequently each pattern occurs in the
            file.
            \item<3-> Sort the patterns by frequency.
            \item<4-> Build a tree from the bottom up, starting with the
            \emph{least frequent} patterns and ending with the most
            frequent patterns.
            \item<5-> Depth of a node in the tree is directly proportional
            to the length of its binary code!
        \end{itemize}
    \end{frame}

    \begin{frame}
        \frametitle{An example}
        feed me more food

        \begin{itemize}
            \item<2-> Find frequencies:

                {\scriptsize
                \begin{tabular}{c|c}
                    f & 2\\
                    e & 4\\
                    d & 2\\
                    \textvisiblespace  & 3\\
                    m & 2\\
                    r & 1\\
                    o & 3\\
                \end{tabular}
                }
            \item <3-> Sort frequencies:

                {\scriptsize
                \begin{tabular}{c|c}
                    r & 1\\
                    f & 2\\
                    d & 2\\
                    m & 2\\
                    \textvisiblespace  & 3\\
                    o & 3\\
                    e & 4\\
                \end{tabular}
                }
        \end{itemize}
    \end{frame}
    \begin{frame}
        \frametitle{An example}
            \begin{tabular}{c|c}
                r & 1\\
                f & 2\\
                d & 2\\
                m & 2\\
                \textvisiblespace  & 3\\
                o & 3\\
                e & 4\\
            \end{tabular}
    \end{frame}
\end{document}
